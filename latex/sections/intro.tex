\documentclass[../main.tex]{subfiles}

\begin{document}
\section{Introduction}\label{sec:intro}
In this work we will study the influences on satellites dynamics starting from better understanding the distribution of mass of the Earth, in order to substitute the common approximation of it as a point-mass with a more complex geopotential model. At the end, we will describe other perturbations of the Earth's gravitational field, such as the influence of the Sun and the Moon, or the effects of atmospheric drag and solar radiation pressure.

In the meantime, we will construct a reference frame where the Newton's laws of motion are valid. But since we will have to know the position of the satellites relative to a ``fixed'' Earth at each step of the integration process, we will have to construct transformations from the former inertial frame to this latter non-inertial frame. In order to do so, we will have to account for the variations on the Earth's axis of rotation as a function of time.

At the end, we will expose the results of the simulations for three kinds of orbits: LEO (Low Earth Orbit), MEO (Medium Earth Orbit) and GEO (Geostationary Earth Orbit).

As a final note, all the proofs of the theorems and propositions done along the document are reported in \cref{sec:proofs_appendix}.
\end{document}